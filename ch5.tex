%
% The Lion's Commentary, file ch5.tex, version 1.6, 24 Nov 2015
%
\section*{Section One}

{\sf Section One contains many of the global declaration
files and the assembly language files.

It also contains a number of files concerned with
system initialisation and process management.}


\se{Two Files}

This chapter is intended to provide a
gentle introduction to the source code
by looking at two files in Section One
which can be isolated reasonably well
from the rest.


The discussion of these files supplements the discussion of Chapter Three
and includes a number of additional
comments regarding the syntax and
semantics of the ``C'' language.


\subsection{The File `malloc.c'}

This file is found on Sheet 25 of the
Source code, and consists of just two
procedures:

\begin{verbatim}
   malloc (2528)   mfree (2556)
\end{verbatim}


These are concerned with the allocation
and subsequent release of two kinds of
memory resources, namely:

\bd
\item[main memory] in units of 32 words (64 bytes);

\item[disk swap area] in units of 256 words (512 bytes).
\ed

For each of these two kinds of
resource, a list of available areas is
maintained within a resource ``map''
(either ``coremap'' or ``swapmap''). A
pointer to the appropriate resource
``map'' is always passed to ``malloc'' and
``mfree'' so that the routines themselves
do not have to know the kind of
resource with which they are dealing.

Each of ``coremap'' and ``swapmap'' is an
array of structures of the type ``map''
as declared at line 2515. This structure consists of two character pointers
i.e. two unsigned integers.

The declarations of ``coremap'' and
``swapmap'' are on lines 0203, 0204.
Here the ``map'' structure is completely
ignored -- a regrettable programming
short-cut which is possible because it
is not detected by the loader. Thus the
actual numbers of list elements in
``coremap'' and ``swapmap'' are ``CMAPSIZ/2''
and ``SMAPSIZ/2'' respectively.


\subsection{Rules for List Maintenance}

\bd
\item[(a)] Each available area is defined
 by its size and relative address
 (reckoned in the units appropriate to the resource);

\item[(b)] The elements of each list are
 arranged at all times in order
 of increasing relative address.
 Care is taken that no two list
 elements represent contiguous
 areas -- the alternative course,
 to merge the two areas into a
 single larger area is always
 taken;
\item[(c)] The whole list can be scanned by
 looking at successive elements
 of the array, starting with the
 first, until an element with a
 zero size is encountered. This
 last element is a ``sentinel''
 which is not part of the list
 proper.
\ed

The above rules provide a complete
specification for ``mfree'', and a
specification for ``malloc'' which is
complete except in one respect:
We need to specify how the
resource allocation is actually
made when there exists more than
one way of performing it.

The method adopted in ``malloc'' is one
known as ``First Fit'' for reasons which
should become obvious.

As an illustration of how the resource
``map'' is maintained, suppose the following
three resource areas were available:

\bi
\item an area of size 15 beginning at
location 47 and ending at location
61;

\item an area of size 13 spanning
addresses 27 to 39 inclusive;

\item an area of size 7 beginning at
location 65.
\ei

\noindent Then the ``map'' would contain:

\begin{center}
\begin{tabular}{ccc}
{\bf Entry} & {\bf Size} & {\bf Address}\\
0 & 13 & 27\\
1 & 15 & 47\\
2 & 7 & 65\\
3 & 0 & ??\\
4 & ?? & ??\\
\end{tabular}
\end{center}

If a request for a space of size 7 were
received, the area would be allocated
starting at location 27, and the ``map''
would become:

\begin{center}
\begin{tabular}{ccc}
{\bf Entry} & {\bf Size} & {\bf Address}\\
0 & 6 & 34\\
1 & 15 & 47\\
2 & 7 & 65\\
3 & 0 & ??\\
4 & ?? & ??\\
\end{tabular}
\end{center}

If the area spanning addresses 40 to 46
inclusive is returned to the available
list, the ``map'' would become:

\begin{center}
\begin{tabular}{ccc}
{\bf Entry} & {\bf Size} & {\bf Address}\\
0 & 28 & 34\\
1 & 7 & 65\\
2 & 0 & ??\\
3 & ?? & ??\\
\end{tabular}
\end{center}

Note how the number of elements has
actually decreased by one because of
amalgamation though the total available
resources have of course increased.

Let us now turn to a consideration of
the actual source code.


\subsection{malloc (2528)}

The body of this procedure consists of
a ``for'' loop to search the ``map'' array
until either:

\bd
\item[(a)] the end of the list of available
 resources is encountered; or

\item[(b)] an area large enough to honour
 the current request is found;
\ed

\bd
\item[2534:] The ``for'' statement initialises
 ``bp'' to point to the first element
	of the resource map. At
 each succeeding iteration ``bp'' is
 incremented to point to the next
 ``map'' structure.
\ed

Note that the continuation condition\\
``bp-$>$m\_size'' is an expression, which becomes zero with the
sentinel is referenced. This
expression could have been written 
equivalently but more transparently as ``bp-$>$m\_size$>$0''.

Note also that no explicit test for the
end of the array is made. (It can be
shown that this latter is not necessary
provided CMAPSIZ, SMAPSIZ $\ge$ 2 * NPROC !)

\bd
\item[2535:] If the list element defines an
 area at least as large as that
 requested, then ...

\item[2536:] Remember the address of the first
 unit of the area;

\item[2537:] Increment the address stored in
 the array element;

\item[2538:] Decrement the size stored in the
 element and compare the result
 with zero (i.e. was it an exact
 fit?);

\item[2539:] In the case of an exact fit, move
 all the remaining list elements
 (up to and including the sentinel) down one place.

Note that ``(bp-l)'' points to the
structure before the one referenced by ``bp'';

\item[2542:] The ``while'' continuation condition does {\bf not}
test the equality of ``(bp-l)-$>$m\_size'' and\\
bm-$>$m\_size !


The value tested is the value assigned to\\
``(bp-$>$m\_size'' copied from ``bp-$>$m\_size''.

(You are forgiven for not
recognising this at once.);


\item[2543:] Return the address of the area.
 This represents the end of the
 procedure and hence very definitely the end of the ``for'' loop.

Note that a value of zero
returned means ``no luck'' This is
based on the assumption that no
valid area can ever begin at
location zero.
\ed

\subsection{mfree (2556)}

This procedure returns the area of size
``size'' at address ``aa'' to the ``resource map''
designated by ``mp''. The body of
the procedure consists of a one line
``for'' statement, followed by a multiline ``if'' statement.

\bd
\item[2564:] The semicolon at the end of this
 line is extremely significant,
 terminating as it does the empty
 statement. (It would aid legibility if this character were moved
 to a line on its own, as is done
 on line 2394.)

Depending on your point of view,
this statement demonstrates
either the power or the obscurity
of the ``C'' language. Try writing
equivalent code to this statement
in another language such as Pascal or PL/1.

Step ``bp'' through the list until
an element is encountered either
with an address greater than the
address of the area being
returned.

 i.e. not ``bp-$>$m\_addr $\le$ a''

\noindent or which indicates the end of the list

 i.e. not ``bp-$>$m\_size != 0'';


\item[2565:] We have now located the element
 in front of which we should
 insert the new list element. The
 question is: Will the list grow
 larger by one element or will
 amalgamation keep the number of
 elements the same or even reduce
 it by one?

If ``bp $>$ mp'' we are not trying to
insert at the beginning of the
list. If\\
(bp-l)-$>$m\_addr+(bp-l)-$>$m\_size==a

\noindent then the area being return abuts
the previous element in the list;

\item [2566:] Increase the size of the previous
list element by the size of the
area being returned;

\item[2567:] Does the area being returned also
 abut the next element of the
 list? If so

\item[2568:] Add the size of the next element
 of the list to the size of the
 previous element;

\item[2569:] Move all the remaining list elements (up to the one containing
 the final zero size) down one
 place.
 
 Note that if the test on line
 2567 fortuitously gives a true
 result when ``bp-$>$m\_size'' is zero
 no harm is done;

\item[2576:] This statement is reached if the
 test on line 2565 failed i.e. the
 area being returned cannot be
 amalgamated with the previous
 element on the list.

 Can it be amalgamated with the
 next element? Note the check that
 the next element is not null;

\item[2579:] Provided the area being returned
 is genuinely non-null (perhaps
 this test should have been made
 sooner?) add a new element to the
 list and push all the remaining
 elements up one place.
\ed


\subsection{In conclusion ...}

The code for these two procedures has
been written very tightly. There is
little, if any, ``fat'' which could be
removed to improve run time efficiency.
However it would be possible to write
these procedures in a more transparent
fashion.


If you feel strongly on this point,
then as an exercise, you should rewrite
``mfree'' to make its function more
easily discernible.

Note also that the correct functioning
of ``malloc'' and ``mfree'' depends on
correct initialisation of ``coremap'' and
``swapmap''. The code to do this occurs
in the procedure ``main'' at lines 1568,
1583.


\subsection{The File `prf.c'}

This file is found on Sheets 23 and 24,
and contains the following procedures:

\begin{verbatim}
    printf (2340)     panic (2416)
    printn (2369)     prdev (2433)
    putchar (2386)    deverror (2447)
\end{verbatim}

The calling relationship between these
procedures is illustrated below:

\newpage
\begin{verbatim}
             panic    deverror
                 |      |
                 |    prdev
                 |      |
                  \    /
                  printf
                    |
                  printn
                    |
                  putchar
\end{verbatim}

\sbs{printf (2340)}

The procedure ``printf'' provides a
direct, unsophisticated low-level,
unbuffered way for the operating system
to send messages to the system console
terminal. It is used during initialisation and to report hardware errors or
the imminent collapse of the system.


(These versions of ``printf'' and
``putchar'' run in kernel mode and are
similar to, but not the same as, the
versions invoked by a ``C'' program which
runs in user mode. The latter versions
of ``printf'' and ``putchar'' live in the
library ``/lib/libc.a''. You may still
find it useful to read the sections
``PRINTF(III)'' and ``PUTCHAR(III)'' of the
UPM at this point.)


\bd
\item[2340:] The programmer must have been
 carried away when he declared all
 the parameters for this procedure. In fact the procedure
 body only contains references to
 ``x1'' and ``fmt''.
\ed


This serves to reveal one of the facts
of ``C'' programming. The rules for
matching parameters in procedure calls
and procedure declarations are not
enforced, not even with respect to the
numbers of parameters.

Parameters are placed on the stack in
{\bf reverse} order. Thus when ``printf'' is
called ``fmt'' will be nearer to the ``top
of stack'' than ``x1'', etc.

\begin{verbatim}
        |   .   |
        ---------
        |   .   |
        ---------
        |   .   |        stack grows down
        ---------
        |   .   |
        ---------
        |  x2   |
        ---------
        |  x1   |
        ---------
        |  fmt  |
        ---------
        |   .   |        top of stack
        ---------
\end{verbatim}

``x1'' has a higher address then ``fmt''
but a lower address then ``x2'', because
stacks grow downwards on the PDP11.

\bd
\item[2341:] ``fmt'' may be interpreted as a
 constant character pointer. This
 declaration is (almost)
 equivalent to

 ``char *fmt;''

The difference is that here the
value of ``fmt'' cannot be changed;

\item[2346:] ``adx'' is set to point to ``x1''.
 The expression ``\&x1'' is the
 address of ``x1''. Note that since
 ``x1'' is a stack location, this
 expression cannot be evaluated at
 compile time.

(Many of the expressions you will
find elsewhere involving the
addresses of variables or arrays
are effective because they {\bf can} be
evaluated at compile or load
time.);

\item[2348:] Extract into the register ``c''
 successive characters from the
 format string;

\item[2349:] If ``c'' is not a `\%' then ...

\item[2350:] If ``c'' is a null character
 (`\verb+\+0'), this indicates the end of
 the format string in the normal
 way, and ``printf'' terminates;

\item[2351:] Otherwise call ``putchar'' to send
 the character to the system console terminal;

\item[2353:] A `\%' character has been seen.
 Get the next character (it had
 better not be the `\verb+\+0'!);

\item[2354:] If this character is a `d' or `l'
 or `o', call ``printn'' passing as
 parameters the value referenced
 by ``adx'' and either the value ``8''
 or ``10'' depending on whether ``c''
 is `o' or not. (The `d' and `l'
 codes are clearly equivalent.)

``printn'' expresses the binary
numbers as a set of digit characters according to the radix 
supplied as the second parameter;

\item[2356:] If the editing character is `s',
 then all but the last character
 of a null terminated string is to
 be sent to the terminal. ``adx''
 should point to a character
 pointer in this case;

\item[2361:] Increment ``adx'' to point to the
 next word in the stack i.e. to
 the next parameter passed to
 ``printf'';

\item[2362:] Go back to line 2347 and continue
 scanning the format string.
 Enthusiasts for structured programming will prefer to replace
 lines 2347 and this by
 ``while (1) \{'' and ``\}''
respectively .
\ed

\sbs{printn (2369)}

This procedure calls itself recursively
in order to generate the required
digits in the required order. It might
be possible to code this procedure more
efficiently but not more completely.
(Anyway, in view of the implementation
of ``putchar'', efficiency is hardly a
consideration here.)

Suppose n = A*b + B where A = ldiv(n,b)
and where B = lrem(n,b) satisfies
$0 \le B < b$. Then in order to display the
value for n, we need to display the
value for A followed by the value for B.


The latter is easy for b = 8 or 10: it
consists of a single character. The
former is easy if A = 0. It is also
easy if ``printn'' is called recursively.
Since A $<$ n, the chain of recursive
calls must terminate.

\bd
\item[2375:] Arithmetic values corresponding
 to digits are conveniently converted to their corresponding
 character representations by the
 addition of the character `0'.
\ed


The procedures ``ldiv'' and ``lrem'' treat
their first parameter as an unsigned
integer (i.e. no sign extension, when a
16 bit value is extended to a 32 bit
value before the actual division operation). They may be found beginning on
lines 1392 and 1400 respectively.


\sbs{putchar (2386)}

This procedure transmits to the system
console the character which was passed
as a parameter.

It illustrates in a small way the basic
features of i/o operations on the PDP11
computer.

\bd
\item[2391:] ``SW'' is defined on line 0166 as
 the value ``0177570''. This is the
kernel address of a read only
processor register which stores
the setting of the console switch
register.

The meaning of the statement is
clear: get the contents at location 0177570 and see if they are
zero. The problem is to express
this in ``C''. The code

if (SW == 0)

would not have conveyed this
meaning. Clearly ``SW'' is a
pointer value which should be
dereferenced. The compiler might
have been changed to accept

if (SW-$>$ == 0)

but as it stands, this is syntactically incorrect. By inventing a
dummy structure, with an element
``integ'' (see line 0175), the programmer has found a satisfactory
solution to his problem.
\ed

Several other examples of this programming device will be found in this
procedure and elsewhere.


In hardware terms, the system console
terminal interface consists of four 16
bit control registers which are given
consecutive addresses on the Unibus
beginning at kernel address 0177560
(see the declaration for ``KL'' on line
0165.) For a description of the formats
and usage of these registers, see
Chapter Twenty-Four of the ``PDP11 Peripherals Handbook''.

In software terms, this interface is
the unnamed structure which is defined
beginning on line 2313, with four elements which name the four control
registers. It does not matter that the
structure is unnamed because it is not
necessary to allocate any instances of
it (the one we are interested in is
essentially predefined, at the address
given by ``KL'').

\bd
\item[2393:] While bit 7 of the transmitter
 status register (``XST'') is off,
 keep doing nothing, because the
 interface is not ready to accept
 another character.
\ed

This is a classic case of ``busy waiting'' where the processor is allowed to
cycle uselessly through a set of
instructions until some externally
defined event occurs. Such waste of
processing power cannot normally be
tolerated but this procedure is only
used in unusual situations.

\bd
\item[2395:] The need for this statement is
 tied up with the statement on
 line 2405;

\item[2397:] Save the current contents of the
 transmitter status register;

\item[2398:] Clear the transmitter status
 register preparatory to sending
 the next character:

\item[2399:] With bit 7 of the control status
 register reset, move the next
 character to be transmitted to
 the transmitter buffer register.
 This initiates the next output
 operation;

\item[2400:] A ``new line'' character needs to
 be accompanied by a ``carriage
 return'' character and this is
 accomplished by a recursive call
 on ``putchar''.

A couple of extra ``delete'' characters are thrown in also to
allow for any delays in completing the carriage return operation
at the terminal;

\item[2405:] This call on ``putchar'' with an
 argument of zero effectively
 results in a re-execution of
 lines 2391 to 2394.

(It is very hard to see why the
programmer chose to use a recursive call here in preference to
simply repeating lines 2393 and
2394, since both code efficiency
and compactness not to mention
clarity seem to have suffered.);

\item[2406:] Restore the contents of the
 transmitter status register. In
 particular if bit 6 was formerly
 set to enable interrupts then
 this resets it.
\ed


\sbs{panic (2419)}

This procedure is called from a number
of locations in the operating system.
(e.g. line 1605). When circumstances
exist under which continued operation
of the system seems undesirable.


UNIX does not profess to be a ``fault
tolerant'' or ``fail soft'' system, and in
many cases the call on ``panic'' can be
interpreted as a fairly unsophisticated
response to a straightforward problem.

However more complicated responses
require additional code, lots of it,
and this is contrary to the general
UNIX philosophy of ``keep it simple''.

\bd
\item[2419:] The reason for this statement is
 given in the comment beginning at
 line 2323;

\item[2420:] ``update'' causes all the large
 block buffers to be written out.
 See Chapter Twenty;

\item[2421:] ``printf'' is called with a format
 string and one parameter, which
 was passed to ``panic'';

\item[2422:] This ``for'' statement defines an
 infinite loop in which the only
 action is a call on the assembly
 language procedure ``idle'' (1284).

``idle'' drops the processor priority to zero, and performs a
``wait''. This is a ``do nothing''
instruction of indefinite duration. It terminates when a
hardware interrupt occurs.

An infinite set of calls on ``idle'' is
better than the execution of a ``halt''
instruction, since any i/o activities
which were under way can be allowed to
complete and the system clock can keep
ticking.

The only way for the operator to
recover from a ``panic'' is to reinitialise the system, (after taking a core
dump, if desired).
\ed


\sbs{prdev (2433), deverror (2447)}

These procedures provide warning messages when errors are occurring in i/o
operations. At this stage, their only
interest is as examples of the use of
``printf''.


\sbs{Included Files}

It will be noted that whereas the file
``malloc.c'' contains no request to
include other files, requests to
include four separate files are
included at the beginning of ``prf.c''.

(The observant reader will note that
these files are presumed to reside one
level higher in the file hierarchy than
``prf.c'' itself.)


The statement on line 2304 is to be
understood as if it were replaced by
the entire contents of the file
``param.h''. This then supplies definitions for the identifiers ``SW'', ``KL''
and ``integ'' which occur in ``putchar''.

We noted earlier that declarations for
``KL'', ``SW'' and ``integ'' occurred on
lines 0165, 0166 and 0175 respectively,
but this would have been meaningless,
if the file ``param.h'' had not been
``included'' in ``prf.c''.

The files ``buf.h'' and ``conf.h'' have
been included to provide declarations
for ``d\_major'', ``d\_minor'', ``b\_dev'' and
``b\_blkno'', which are used in ``prdev''
and ``deverror''.

The reason for the inclusion of the
fourth file, ``seg.h'', is a little
harder to find. In fact it is not
necessary as the code stands, and the
author owes his readers an apology. In
editing the source code, it seemed like
a good idea to move the declaration for
``integ'' from ``seg.h'' to ``param.h''.
Q.E.D.

Note that the variable ``panicstr''
(2328) is also global but since it is
not referenced outside ``prf.c'', its
declaration has not been placed in any
``.h'' file.
