%
% The Lion's Commentary, file ch19.tex, version 1.4, 16 May 1994
%
\se{File  Directories and Directory Files}

As we have seen, much important  information  about
individual files is contained in the ``inode'' tables.   If  the
file  is currently accessible, or being
accessed, the relevant  information  is
held  in  the core ``inode'' table.  If a
file is on  disk  (more  generally,  on
some  file  system volume) and is not
currently accessible, then the relevant
``inode''  table  is  the one recorded on
the disk (file system volume).


Notably absent from the  ``inode''  table
is any information regarding the ``name''
of the file.  This  is  stored  in  the
directory files.

\sbs{The Directory Data Structure}

Each file must have at least one  name.
A file may have more than one  distinct
name, but the  same  name  may  not  be
shared  by  two  distinct  files,  i.e.
each name must define a unique file.


A name may be multipart.  When written,
the parts or components of the name are
separated by slashes (``/'').  The  order
of components within a name is significant i.e. 
``a/b/c''  is  different  from ``a/c/b''.

If file  names  are  divided  into  two
parts:  an initial part or ``stem'' and a
final part or ``ending'', then two  files
whose  names  have  identical stems are
usually relate in some way.  They  may
reside  on  the  same  disk,  they  may
belong to the same user, etc.
Users make initial reference  to  files
by  quoting  the file name, e.g. in the
``open''  system  call.    An   important
operating  system function is to decode
the name into the corresponding ``inode''
entry.   To  do  this, UNIX creates and
maintains a directory  data  structure.
This   structure  is  equivalent  to  a
directed graph with named edges.


In its purest form, the graph is a tree
i.e.  it  has  a single root node, with
exactly one path between the  root  and
any  node.  More  commonly in UNIX (but
not so commonly in other operating systems) the graph is a lattice  which may
be obtained from a tree  by  coalescing
one or more groups of leaves.

In this case, while there is still only
one path between the root and any interior node, there may be more  than  one
path  between  the  root  and  a  leaf.
Leaves are nodes without successors and
correspond to  data  files.  Interior
nodes are  nodes  with  successors  and
correspond to directory files.

The name for a file  is  obtained  from
the  names  of  the  edges  of the path
between   the   root   and   the   node
corresponding  to  the file.  (For this
reason, the name is often  referred  to
as  a ``pathname''.) If there are several
paths, then the file has several names.


\sbs{Directory Files}

A directory file is  in  many  respects
indistinguishable  from a non-directory
file.  However it contains  information
which  is  used in locating other files
and hence its  contents  are  carefully
protected,  and  are manipulated by the
operating system alone.


In  every  file,  the  information   is
stored  as  one  or  more 512 character
blocks.  Each block of a directory file
is  divided  into  32  *  16  character
structures.  Each structure consists of
a 16 bit ``inode'' table pointer and a 14
character name.  The ``inode'' pointer is
to  the  ``inode'' table on the same disk
or file  system  volume  as  the  files
which  the  directory references. (More
on this later.)  An  ``inode''  value  of
zero defines a null entry in the directory.

The procedures which  reference  directories are:

\begin{verbatim}
   namei  (7518) search directory
   link   (5909) create alternate name
   wdir   (7477) write directory entry
   unlink (3510) delete name
\end{verbatim}

\sbs{namei (7518)}

\bd
\item[7531:] ``u.u\_cdir'' defines the ``inode'' of
      a process's current directory.  A
      process  inherits  its   parent's
      current    directory   at   birth
      (``newproc'', 1883).   The  current
      directory  may  be  changed using
      the ``chdir'' (3538) system call;
\item[7532:] Note that ``func'' is  a  parameter
      to  ``namei''  and is always either
      ``uchar'' (7689) or ``schar'' (7679);
\item[7534:] ``iget'' (7276) is called to:
\bi
\item wait  until  such  time  as   the
``inode''  corresponding to ``dp'' is
no longer locked;

\item check that  the  associated  file system is still mounted;
\item increment the reference count;
\item lock the ``inode'';
\ei

\item[7535:] Multiple slashes are  acceptable! (i.e.\\
 ``////a///b/'' is the same as ``/a/b'');

\item[7537:] Any attempt to replace or  delete
      the  current working directory or
      the  root  directory  is  bounced
      immediately!

\item[7542:] The  label  ``cloop''   marks   the
      beginning  of a program loop that
      extends to line 7667.  Each cycle
      analyses a component of the pathname (i.e. a string terminated by
      a  null  character or one or more
      slashes).  Note that a  name  may
      be  constructed  from  many  different characters (7571);

\item[7550:] The end of the pathname has  been
      reached  (successfully).   Return
      the current value of ``dp'';

\item[7563:] ``search''  permission  for  directories  is  coded in the same way
      as ``execute'' permission for other files;

\item[7570:] Copy the name into a more  accessible  location before attempting
      to  match  it  with  a  directory
      entry.    Note  that  a  name  of
      greater than ``DIRSIZ''  characters
      is truncated;

\item[7589:] ``u.u\_count'' is set to the  number
      of entries in the directory;

\item[7592:] The  label  ``eloop''   marks   the
      beginning of a program loop which
      extends to line 7647. Each cycle
      of the loop handles a single
      directory entry;

\item[7600:] If the directory has been
      searched (linearly!) without
      matching the supplied pathname
      component, then there must be an
      error unless:

\bd
     \item[(a)] this is the last component of
      the pathname, i.e. ``c==`\verb+\+0' '';

      \item[(b)] the file is to be created,
      i.e. ``flag == 1''; and

      \item[(c)] the user program has ``write''
      permission for the directory;
\ed

\item[7606:] Record the ``inode'' address for
      the directory for the new file in
      ``u.u\_pdir'';

\item[7607:] If a suitable slot for a new
      directory entry has previously
      been encountered (7642), store
      the value in ``u.u\_offset[1]'';
      else set the ``IUPD'' flag for the
      ``dp'' designated ``inode'' (but why?);

\item[7622:] When appropriate, read a new
      block from the directory file
      (note the use of ``bread'') (why
      not ``breada''?), after carefully
      releasing any previously held
      buffer;

\item[7636:] Copy the eight words of the
      directory entry into the array
      ``u.u\_dent''. The reason for copying before comparing is obscure!
      Can this actually be more efficient? (The reason for copying
      the whole directory at all is
      rather perplexing to the author
      of these notes.);

\item[7645:] This comparison makes efficient
      use of a single character pointer
      register variable, ``cp''. The
      loop would be even more efficient
      if word by word comparison were
      used;

\item[7647:] The ``eloop'' cycle is terminated
      by one of:

\begin{verbatim}
   return(NULL);    (7610)
   goto out;        (7605, 7613)
\end{verbatim}

\noindent a successful match  so  that  the
branch  to  ``eloop'' (7647) is not
taken;

\item [7657:] If the  name  is  to  be  deleted
(``flag==2''),  if the pathname has
been completed, and if  the  user
program has ``write'' access to the
directory, then return a  pointer
to the directory ``inode'';

\item[7662:] Save  the  device  identity  temporarily (why not in the register
``c''?) and call ``iput''  (7344)  to
unlock  ``dp'',  to  decrement  the
reference count on  ``dp''  and  to
perform  any  consequent processlng;

\item[7664:] Revalidate ``dp'' to point  to  the
      ``inode'' for the next level file;

\item[7665:] ``dp==NULL''   shouldn't    happen,
since the directory says the file
exists!  However  ``inode''   table
overflows   and  i/o  errors  can
occur,  and  sometimes  the  file
system  may  be left in an inconsistent  state  after  a   system
crash.
\ed

\sbs{Some Comments}


``namei'' is a key procedure which  would
seem  to  have been written very early,
to have been  thoroughly  debugged  and
then  to  have  been  left  essentially
unchanged.   The   interface    between
``namei''  and  the rest of the system is
rather complex,  and  for  that  reason
alone,  it  would not win the prize for
``Procedure of the Year''.


``namei'' is  called  thirteen  times  by
twelve different procedures:

\begin{tabular}{llll}\\
{\bf line} & {\bf routine} & \multicolumn{2}{l}{\bf parameters} \\ \hline
3034 & exec & uchar & 0 \\
3543 & chdir & uchar & 0 \\
5770 & open & uchar & 0 \\
5914 & link & uchar & 0 \\
6033 & stat & uchar & 0 \\
6097 & smount & uchar & 0 \\
6186 & getmdev & uchar & 0 \\
6976 & owner & uchar & 0 \\
\\
5786 & creat & uchar & 1 \\
5928 & link & uchar & 1 \\
5958 & mknod & uchar & 1 \\
\\
3515 & unlink & uchar & 2 \\
\\
4101 & core & schar & 1 \\
\end{tabular}

\bigskip

\noindent It will be seen that:
\bd
\item[(a)] there are two calls from ``link'';

\item [(b)] the calls can  be  divided  into
four  categories,  of  which the
first is by far the largest;

\item[(c)] the  last  two  categories  have
only one representative each;

\item[(d)] in particular, there is only one
call   involving   the   routine
``schar'', which is always  for  a
file  called  ``core''.  (If  this
case were handled as  a  special
case   e.g.   where  the  second
parameter  had  the  value  ``3'',
then  the  ``uchar''s  and ``schar''
could be eliminated.)
\ed


\noindent ``namei'' may terminate in a  variety  of
ways:

\bd
\item[(a)]  if there has been an error, then
     a  ``NULL''  value is returned and
     the variable ``u.u\_error'' is set.

(Most errors result in a  branch
to  the  label  ``out''  (7669) so
that reference  counts  for  the
inodes are properly maintained
(7670). This is not necessary if
the  failure  occurs  in  ``iget''
(7664).);



\item[(b)]  if ``flag==2'' (i.e. the  call  is
     from    ``unlink''),   the   value
     returned   (in    normal    circumstances)    is   an   ``inode''
     pointer for the parent directory
     of the named file (7660);

\item[(c)]  if ``flag==1'' (i.e. the  call  is
     from   ``creat''   or   ``link''  or
     ``mknod'', and a  file  is  to  be
     created  if  it does not already
     exist) and  if  the  named  file
     does  not  exist,  then a ``NULL''
     value  is  returned  (7610).  In
     this   case  a  pointer  to  the
     ``inode'' for the directory  which
     will  point  to the new file, is
     left in ``u.u\_pdir'' (7606). (Note
     also    that   in   this   case,
     ``u.u\_offset''  is  left  pointing
     either  at  an  empty  directory
     entry  or  at  the  end  of  the
     directory file.);

\item[(d)] if  in the remaining cases,  the
file  exists, an ``inode'' pointer
for the file is returned (7551).
The  ``inode''  is  locked and the
reference count has been  incremented.   A  call  to  ``iput'' is
needed subsequently to undo both
these side effects.
\ed



\sbs{link (5909)}

This procedure implements a system call
which enters a new name for an existing
file  into  the  directory   structure.
Arguments  to  the  procedure  are  the
existing and the new names of the file;

\bd
\item[5914:] Look up the existing file name;

\item[5917:] If the file already has 127  different names, quit in disgust;

\item[5921:] If the existing file turns out to
      be  a  directory,  then  only the
      super-user may rename it;

\item[5926:] Unlock the existing file  ``inode''
      This  is  locked  when  the first
      call on ``namei''  does  an  ``iget''
      (7534, 7664).

Under what conditions  would  the
failure  to  unlock  the  ``inode''
here be disastrous?  The  chances
that the existing file would be a
directory  encountered   in   the
search  for  the  new  name would
seem slight, if  not  impossible.
Most  probably  the relevant circumstance is where the system  is
attempting  to recreate an alternative file name or alias,  which
{\bf already} exists;

\item[5927:] Search  the  directory  for   the
      second  name,  with the intention
      of creating a new entry;

\item[5930:] There is an  existing  file  with
      the second name;

\item[5935:] ``u.u\_pdir'' is set as a side effect
of  the  call  on ``namei'' (5928).
Check that the directory  resides
on the same device as the file;

\item[5940:] Write a new directory entry  (see
      below);

\item[5941:] Increase the ``link'' count for the
      file.
\ed


\sbs{wdir (7477)}

This procedure enters a new name into a
directory.   It  is  called  by  ``link''
(5940)  and  ``maknode''  (7467)  with  a
pointer  to a (core) ``inode'' as parameter.


The sixteen characters of the directory
entry  are  copied  into  the structure
``u.u\_dent'', and written from there into
the directory file. (Note that the previous content of ``u.u\_dent''  will  have
been  the name of the last entry in the
directory file.)


The procedure assumes that  the  directory  file  has  already been searched,
that the ``inode'' for the dlrectory file
has already been allocated and that the
values of ``u.u\_offset''  have  been  set
appropriately.

\sbs{maknode (7455)}

This procedure is  called  from  ``core''
(4105),   ``creat''  (5790)  and  ``mknod''
(5966),  after  a  previous   call   on
``namei'' with a second parameter of one,
has revealed that no file of the specified name existed.


\sbs{unlink (3510)}

This procedure implements a system call
which  deletes  a  file  name  from the
directory structure.  (When all  references  to  a file are deleted, the file
itself will be deleted.)

\bd
\item[3515:] Search for a file with the specified
 name,  and  if  it  exists,
return a pointer to  the  ``inode''
of  the  immediate  parent directory;


\item[3518:] Unlock the parent directory;

\item[3519:] Get an  ``inode''  pointer  to  the
file itself;

\item[3522:] Unlinking directories is  forbidden, except for super-users;

\item[3528:] Rewrite the directory entry  with
      the ``inode'' value set to zero;

\item[3529:] Decrement the ``link'' count.
\ed


Note that there is no attempt to reduce
the size of a directory below its ``high
water'' mark.


\sbs{mknod (5952)}

This procedure, which implements a system call
of the same name, is only executable by the super-user. As explained
in  the Section ``MKNOD(II)'' of the UPM,
this system  call  is  used  to  create
``inodes'' for special files.

``mknod''  also  solves  the  problem  of
``where  do directories come from''?  The
second parameter passed to  ``mknod''  is
used,  without modification or restriction to set ``i\_mode''.
(Compare  ``creat'' (5790)  and  ``chmod''  (3569)).  This is
the only way an ``inode'' can get flagged
as a directory, for instance.

In  such  cases,  the  third  parameter
passed  to  ``mknod'' {\bf must} be zero.  This
value is copied into ``i\_addr[0]'' (as is
appropriate for special files), and, if
non-zero, will be accepted uncritically
by  ``bmap''  (6447). It might be prudent
to insert a test

\begin{verbatim}
   if (ip->i_mode & (IFCHR & IFBLK) != 0)
\end{verbatim}

\noindent before  line  5969,  rather  than  rely
indefinitely  on  the  infallibility of
the super-user.


\sbs{access (6746)}

This  procedure  is  called  by  ``exec''
(3041),  ``chdir'' (3552), ``core'' (4109),
``open1'' (5815,  5817),  ``namei''  (7563,
7664,  7658) to check access permission
to  a  file.  The   second   parameter,
``mode'',  is  equal  to  one of ``IEXEC'',
``IWRITE'' and ``IREAD'', with octal values
of 0100, 0200 and 0400 respectively.

\bd
\item[6753:] ``write'' permission is  denied  if
      the  file  is  on  a  file system
      volume which has been mounted  as
      ``read  only''  or  if  the file is
      functioning as the  text  segment
      for an executing program;

\item[6763:] the suer-user may not execute  a
file unless it is ``executable'' in
at least one of the  three  ``permission''  groups.  In  any  other
situation he  is  always  allowed
access;

\item[6769:] If the user is not the  owner  of
      the  file, shift ``m'' three places
      to the right so that  group  permissions will be operative ... If
      the groups don't match, shift ``m''
      again;

\item[6774:] Compare ``m'' and the  access  permissions.
\ed


Note that there is an anomaly  here  in
that  if  a  file has a ``mode'' of 0077,
the owner cannot reference it  at  all,
but  everyone  else can. This situation
could  be  changed  satisfactorily   by
inserting a statement

\begin{verbatim}
   m =|  (m |  (m >> 3)) >> 3;
\end{verbatim}

\noindent after line 6752,  and  replacing  lines
6764, 6765 by

\begin{verbatim}
   if (m & IEXEC && (m & ip->i_mode) == 0)
\end{verbatim}
